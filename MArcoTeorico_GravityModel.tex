% Options for packages loaded elsewhere
\PassOptionsToPackage{unicode}{hyperref}
\PassOptionsToPackage{hyphens}{url}
\PassOptionsToPackage{dvipsnames,svgnames*,x11names*}{xcolor}
%
\documentclass[
  spanish,
  twocolumn]{article}
\usepackage{lmodern}
\usepackage{amssymb,amsmath}
\usepackage{ifxetex,ifluatex}
\ifnum 0\ifxetex 1\fi\ifluatex 1\fi=0 % if pdftex
  \usepackage[T1]{fontenc}
  \usepackage[utf8]{inputenc}
  \usepackage{textcomp} % provide euro and other symbols
\else % if luatex or xetex
  \usepackage{unicode-math}
  \defaultfontfeatures{Scale=MatchLowercase}
  \defaultfontfeatures[\rmfamily]{Ligatures=TeX,Scale=1}
\fi
% Use upquote if available, for straight quotes in verbatim environments
\IfFileExists{upquote.sty}{\usepackage{upquote}}{}
\IfFileExists{microtype.sty}{% use microtype if available
  \usepackage[]{microtype}
  \UseMicrotypeSet[protrusion]{basicmath} % disable protrusion for tt fonts
}{}
\makeatletter
\@ifundefined{KOMAClassName}{% if non-KOMA class
  \IfFileExists{parskip.sty}{%
    \usepackage{parskip}
  }{% else
    \setlength{\parindent}{0pt}
    \setlength{\parskip}{6pt plus 2pt minus 1pt}}
}{% if KOMA class
  \KOMAoptions{parskip=half}}
\makeatother
\usepackage{xcolor}
\IfFileExists{xurl.sty}{\usepackage{xurl}}{} % add URL line breaks if available
\IfFileExists{bookmark.sty}{\usepackage{bookmark}}{\usepackage{hyperref}}
\hypersetup{
  pdftitle={MODELO GRAVITACIONAL DE COMERCIO REGIONAL BILATERAL},
  pdflang={es},
  pdfkeywords={``first keyword'' ``second keyword''},
  colorlinks=true,
  linkcolor=blue,
  filecolor=Maroon,
  citecolor=Blue,
  urlcolor=Blue,
  pdfcreator={LaTeX via pandoc}}
\urlstyle{same} % disable monospaced font for URLs
\usepackage[margin=1in]{geometry}
\usepackage{graphicx}
\makeatletter
\def\maxwidth{\ifdim\Gin@nat@width>\linewidth\linewidth\else\Gin@nat@width\fi}
\def\maxheight{\ifdim\Gin@nat@height>\textheight\textheight\else\Gin@nat@height\fi}
\makeatother
% Scale images if necessary, so that they will not overflow the page
% margins by default, and it is still possible to overwrite the defaults
% using explicit options in \includegraphics[width, height, ...]{}
\setkeys{Gin}{width=\maxwidth,height=\maxheight,keepaspectratio}
% Set default figure placement to htbp
\makeatletter
\def\fps@figure{htbp}
\makeatother
\setlength{\emergencystretch}{3em} % prevent overfull lines
\providecommand{\tightlist}{%
  \setlength{\itemsep}{0pt}\setlength{\parskip}{0pt}}
\setcounter{secnumdepth}{-\maxdimen} % remove section numbering
\usepackage{pgf,tikz}
\ifxetex
  % Load polyglossia as late as possible: uses bidi with RTL langages (e.g. Hebrew, Arabic)
  \usepackage{polyglossia}
  \setmainlanguage[]{spanish}
\else
  \usepackage[shorthands=off,main=spanish]{babel}
\fi
\usepackage[]{biblatex}
\addbibresource{./references.bib}

\title{\textbf{MODELO GRAVITACIONAL DE COMERCIO REGIONAL BILATERAL}}
\usepackage{etoolbox}
\makeatletter
\providecommand{\subtitle}[1]{% add subtitle to \maketitle
  \apptocmd{\@title}{\par {\large #1 \par}}{}{}
}
\makeatother
\subtitle{EII-805 Metodología de la Investigación TALLER 04}
\author{Manuel Ayala\(^a\)\\
\(^a\)Pontificia Universidad Católica de Valparaíso}
\date{September 29, 2020}

\begin{document}
\maketitle
\begin{abstract}
El presente taller 04 esta basado en el desarrollo del trabajo de
investigación con marco en las ciencias económicas y bajo el contexto
regional latinoamericano, a partir de un trabajo de investigación que
tiene por objetivo principal identificar las razones que explican el
comercio bilateral entre pares de paises y sus economías a través de un
modelo gravitacional. Este taller da cuenta de las metodologías
utilizadas para la busqued, extracción y recopilación de información qeu
fundamenten lineamientos de marco teórico y conceptual del proyecto de
investigación, considerando para esto bases de datos públicas y análisis
bibliométrico de la situación actual sobre el tema de investigación
abordado e incluyendo potenciales trabajos futuros.

\textbf{Keywords:} Modelos Gravitacionales, Econometría Espacial,
Comercio Latinoamericano.
\end{abstract}

\hypertarget{introducciuxf3n}{%
\subsection{Introducción}\label{introducciuxf3n}}

Divulgar resultados de un estudio, publicar a la comunidad, contribuir
al desarrollo con nuevo conocimiento, son cualidades que quedan
plasmadas en un artículo
científico\autocite{vallina-hernandez_gravity_2020}, el cual será al
menos, leído por una revista para su aceptación a publicación, este
proceso tiene en sus inicios etapas de indagación científica que debe
ser profunda metódica y organizada de acuerdo al tema de investigación,
el presente taller 04 se construye a partir de estas esta búsqueda
organizada de información que lineará el marco teórico y conceptual de
la investigación, mostrando las actividades realizadas para justificar
la realización de la idea de investigación.

\hypertarget{discuciuxf3n-bibliogruxe1fica}{%
\subsection{Discución
bibliográfica}\label{discuciuxf3n-bibliogruxe1fica}}

Se realizó revisión de literatura desde dos distintos enfoques,
considerando primero búsqueda y organización de información como una
nube de ideas generales bajo la técnica de mapeo y las posibles teorías
científicas que podrían participar en la intención de investigación
científica planteada, luego de esta búsqueda amplia y revisión de la
literatura encontrada, se acotó la información a tres teorías
específicas de desarrollo, considerando que unen dos conceptos generales
``espacial'' y ``tiempo'' en el contexto econométrico. Esta recopilación
de literatura se indexó a través de archivos digitales en dos sistemas
contenedores, a través del software Zotero incluido en google Chrome y a
través de carpetas organizadas y contenidas en carpetas y subcarpetas
organizadas como técnica indexada. Estas técnicas se utilizaron
complementarias para la organización de la búsqueda organizada y se
alimentaron a través de fuentes abiertas de información, primeramente
con motores de búsqueda en internet como google, dada la intención de
obtener una visión general del estudio de interés de características
transdiciplinario por el concepto teórico modelo econométrico espacial,
para luego con un norte claro de dirección de búsqueda, estrechar la
información a un carácter específico con otra forma de revisión a través
de literatura científica citada en el artículo bajo estudio y
recopilando información complementaria que entregara una visión del
estado actual de la cituación científica respecto al tema de interés,
utilizando para esto a motores especializados en estudios científicos
como Google Scholar y Web of Science, de estos se rescató listado de
texto formato *.txt ocupando keywords para rescatar el estado actual y
relevante del problema plantaedo como investigación. Como tercera línea
de información necesaria para fundar la investigación se realizó
indagación de datos que poudieran sustentar el modelo de investigación,
esto se realizó considerando los organismos gubernamentales que cuentan
con bases de datos validas para fundar y resolver el modelo planteado
como investigación. A continuación se presenta imagen de la busqueda
indexada.

\hypertarget{marco-conceptual-y-teuxf3rico}{%
\subsection{Marco Conceptual y
Teórico}\label{marco-conceptual-y-teuxf3rico}}

``La interacción espacial se podría caracterizar como el flujo
resultante del movimiento a escala espacio temporal de mediano o largo
plazo de elementos, los cuales poseen un punto de origen y uno de
destino, ambos con una localización''\autocite{miranda_aplicacion_2018}.
De esta definición se desprende la idea de interacción entre agentes
como por ejemplo países, los cuales tienen ubicaciones espaciales y se
relacionan con un flujo en el tiempo a través de actividades variadas
entre ellos y esta interacción bajo un carácter espacial en una
geografía específica y condicionante. Este concepto sustenta
metodologías de estudio que consideran la interacción entre agentes, su
identidad regional y sus flujos de información variada en intensidad y
dirección, con la intensión de modelar estos flujos de comunicación en
el tiempo, se han estudiado modelos que caractericen estas
interacciones, de acuerdo a Miranda y Jara (2018)
\autocite{miranda_aplicacion_2018}, a partir de los años 1930 W.J.
Reilly, plantea uno de los primeros modelos de comercio basados en la
Ley de Gravitación Universal adaptada a realidades sociales y en este
caso en particular al contexto económico comercial entre agentes
participantes de la interacción de flujos, planteando un modelo de la
forma:

\[
F_{ij} = k (P_i P_j/DIST_{ij})
\]

Donde el \(F_{ij}\) representa el flujo total entre las zonas \(i\) y
\(j\), \(k\) es una constante determinada por el ajuste del modelo,
\(P_i\) y \(P_j\) son las masas de cada zona o región y por último
\(DIST_{ij}\) es la distancia que separa las dos zonas a estudiar.

Desde el punto de vista econométrico esta metodología gravitacional y de
la mano con nuevos algoritmos y \emph{software} permitíeron el cálculo y
análisis comerciales, considerando y modelando en conjunto a variables
espacios geográficos y temporales. En la actualidad se considera como
una metodología ampliamente utilizada para la predicción de los flujos
comerciales internacionales \autocite{vallina-hernandez_gravity_2020}.

Una definición econométrica de modelo gravitatorio más sofisticada es la
registrada por Mátyás (1997) \autocite{matyas_proper_1997} la cual
considera en su planteamiento el flujo comercial bilateral en el tiempo
y espacio geográfico.

\[ 
\tiny ln EXP_{ijt}= \alpha_i+\lambda_t+\gamma_j+\beta_1 ln Y_{it}+\beta_2 ln Y_{jt}+\beta_3 DIST_{ij}+ \ldots+u_{ijt}
\]

Donde: \(EXP_{ijt}\) es el volumen de comercio entre el país \(i\) con
el país \(j\) al tiempo \(t\); \(Y_{it}\) es el Producto Interno Bruto
del país \(i\) al tiempo \(t\); \(DIST_{ij}\) corresponde a la distancia
entre países;\((\alpha_i,\gamma_j)\) efectos en los países \((i,j)\);
\(\lambda_t\) es el efecto del tiempo de comercio en el tiempo \(t\) y
\$u\_\{ijt\} es el término de error del modelo.

El modelo identifica el volumen de comercio bilateral entre países con
un contexto regional de América Latina y el flujo comercial
intra-regional de exportaciones
\autocite{vallina-hernandez_gravity_2020} a partir de información
contenida en diferentes fuentes como el Banco Mundial, el Fondo
Monetario Internacional, COMTRADE de Naciones Unidas entre otros.

\hypertarget{resultados-del-anuxe1lisis-bibliomuxe9trico}{%
\subsection{Resultados del análisis
bibliométrico}\label{resultados-del-anuxe1lisis-bibliomuxe9trico}}

Se realizó a partir de las capacidades de los algoritmos generados en el
lenguaje R ``coword'' y ``Bibliometrix'', un análisis bibliométrico del
estado de la situación científica actual de las palabras claves
keywords: \textbf{Econometric Modeling + spatial}, búsqueda de
literatura realizada en la base artículos científicos Web of Science
(WoS). La literatura relacionada encontrada de acuerdo a frase clave y
considerando solo los años 2019 y 2020 de producción científica, se
determinaron 384 artículos. A partir de esta base de datos obtenida en
la WoS se extrajeron los datos en formato \emph{PlainText} y modalidad
\emph{Full Report}, objeto de ingresar la información a análisis
bibliométrico en los algorítmos Coword y Bibliometrix.

El resultado del análisis estadístico identificó como palabras más
utilizadas a ``Spatial'' y ``China'', de acuerdo a la siguiente nube de
palabras e histograma de frecuencias siguientes figuras 2 y 3.

\printbibliography

\end{document}
